\documentclass[10pt, a4paper]{article}

\usepackage[english]{babel}
\usepackage[latin1]{inputenc}
\usepackage{float}
\usepackage{listings}


\begin{document}
\lstset{language=C,
        breaklines=true
        }
\section{Test clock settings}
\begin{table}[H]
\centering
\begin{tabular}{l l l l l l}
Clock Settings [MHz] & 2 & 12 & 33 & 48 & 66 \\
\hline
PLL Frequency [MHz]  & 48 & 48  & 66 & 96 & 132\\
\hline
PLLMUL & 3 & 3 & 10 & 15 & 10 \\
\hline
PLLDIV& 0 & 0 & 1 & 1 & 1 \\
\hline
PLL\_FREQ [MHz]& 48 & 48 & 66 & 96 & 132 \\
\hline
PLL\_DIV2& 1 & 1 & 1 & 1 & 0 \\
\hline
PBADIV& 1 & 1 & 1& 1 & 1\\
\hline
PBASEL& 0 & 0 & 0 & 0 & 0\\
\hline
PBBDIV& 1 & 1 & 1 & 1 & 1 \\
\hline
PBBSEL& 0 & 0 & 0 & 0 & 0\\
\hline
HSBDIV& 1 & 1 & 1 & 1 & 1\\
\hline
HSBSEL (CPUSEL)& 3 & 1 & 0 & 0 & 0\\
\hline
Works as expected?& Yes & Yes & Yes & Yes & \\
\hline
Approximate blinking freq. [Hz]& 1/4 & 1  & 3 & 4 & 5 \\
\hline
Measured power consumption [mW] & 515 & 545 & 620 & 680 & 730 \\
\hline

\end{tabular}
\end{table}


\textbf{Conclusion}\newline
Power consumtion and blink frequency seems to be increasing proportionally to the clock frequency.
\section{Using support functions}
\textbf{a.} The Slow Clock is called $RCSYS$ in the header file. It is called
from the function $pcl\_configure\_clocks\_rcsys(pc\_freq\_param\_t *param)$. \newline
\textbf{b.} The struct has definition
\begin{lstlisting}{frame=single}

typedef struct
{
  //! Main clock source selection (input argument).
  pcl_mainclk_t main_clk_src;

  //! Target CPU frequency (input/output argument).
  unsigned long cpu_f;

  //! Target PBA frequency (input/output argument).
  unsigned long pba_f;

  //! Target PBB frequency (input/output argument).
  unsigned long pbb_f;

  //! Target PBC frequency (input/output argument).
  unsigned long pbc_f;

  //! Oscillator 0's external crystal(or external clock) frequency (board dependant) (input argument).
  unsigned long osc0_f;

  //! Oscillator 0's external crystal(or external clock) startup time: AVR32_PM_OSCCTRL0_STARTUP_x_RCOSC (input argument).
  unsigned long osc0_startup;

  //! DFLL target frequency (input/output argument) (NOTE: the bigger, the most stable the frequency)
  unsigned long dfll_f;

  //! Other parameters that might be necessary depending on the device (implementation-dependent).
  // For the UC3L DFLL setup, this parameter should be pointing to a structure of
  // type (scif_gclk_opt_t *).
  void *pextra_params;
} pcl_freq_param_t;

\end{lstlisting}


\end{document}
