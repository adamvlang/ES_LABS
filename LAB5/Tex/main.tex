\documentclass[10pt, a4paper]{article}

\usepackage[english]{babel}
\usepackage[latin1]{inputenc}
\usepackage{float}
\usepackage{listings}
\usepackage{amsmath}


\begin{document}
\lstset{language=C,
        breaklines=true
        }
\section{Test clock settings}
\begin{table}[H]
\centering
\begin{tabular}{l | l l l l l}
Clock Settings [MHz] & 2 & 12 & 33 & 48 & 66 \\
\hline
PLL Frequency [MHz]  & 48 & 48  & 66 & 96 & 132\\
\hline
PLLMUL & 3 & 3 & 10 & 15 & 10 \\
\hline
PLLDIV& 0 & 0 & 1 & 1 & 1 \\
\hline
PLL\_FREQ [MHz]& 48 & 48 & 66 & 96 & 132 \\
\hline
PLL\_DIV2& 1 & 1 & 1 & 1 & 0 \\
\hline
PBADIV& 1 & 1 & 1& 1 & 1\\
\hline
PBASEL& 0 & 0 & 0 & 0 & 0\\
\hline
PBBDIV& 1 & 1 & 1 & 1 & 1 \\
\hline
PBBSEL& 0 & 0 & 0 & 0 & 0\\
\hline
HSBDIV& 1 & 1 & 1 & 1 & 1\\
\hline
HSBSEL (CPUSEL)& 3 & 1 & 0 & 0 & 0\\
\hline
Works as expected?& Yes & Yes & Yes & Yes & \\
\hline
Approximate blinking freq. [Hz]& 1/4 & 1  & 3 & 4 & 5 \\
\hline
Measured power consumption [mW] & 515 & 545 & 620 & 680 & 730 \\
\hline

\end{tabular}
\end{table}


\textbf{Conclusion}\newline
Power consumtion and blink frequency seems to be increasing proportionally to the clock frequency.
\section{Using support functions}
\textbf{a.} The Slow Clock is called $RCSYS$ in the header file. It is called
from the function $pcl\_configure\_clocks\_rcsys(pc\_freq\_param\_t *param)$. \newline
\textbf{b.} The struct has definition
\begin{lstlisting}{frame=single}

typedef struct
{
  //! CPU frequency (input/output argument).
  unsigned long cpu_f;

  //! PBA frequency (input/output argument).
  unsigned long pba_f;

  //! Oscillator 0's external crystal(or external clock) frequency (board dependant) (input argument).
  unsigned long osc0_f;

  //! Oscillator 0's external crystal(or external clock) startup time: AVR32_PM_OSCCTRL0_STARTUP_x_RCOSC (input argument).
  unsigned long osc0_startup;
} pm_freq_param_t;

\end{lstlisting}
, where $osc0\_f$ is set to $FOSC0$ and $osc0\_startup$ is set to $OSC0\_STARTUP$.
\section{Real Time Clock}
\textbf{d.} When using the 32kHz crystal, $RTC\_PSEL\_32KHZ\_1HZ = 14$ gives RTC
at 1 Hz.
\newline
\textbf{e.} Using the equation in the header file $rtc.h$ to choose $PSEL$, one
can calculate the desired frequency. For a 2.5 second interrupt frequency, using
2 Hz and interrupting at every 5 ticks would give the correct interrupt time.
Choosing $PSEL$ is done by using
\begin{equation}
    PSEL = \frac{\log{\frac{f_{osc}} {f_{rtc}}}} {\log{2}}-1.
\end{equation}
\newline
\textbf{g.} The LED switches at the same frequency. Power consumtion is lower
than before. The RTC is not dependant on CPU speed, which makes it more general.
It is also more exact.
\newline
\textbf{b.} Power consumtion in described in table below. Note that the display
is on.
\begin{table}[H]
    \centering
    \begin{tabular}{l | l l}
        Clock speed [MHz] & Power consumtion sleep [mW] & Power consumtion
        software [mW]\\
        \hline
        12 & 560 & 585 \\
        \hline
        33 & 580 & 645 \\
        \hline
        66 & 610 & 715 \\
    \end{tabular}
\end{table}
\textbf{c./d.} The following table displays modes that are working and what
power consumption they have.
\begin{table}[H]
    \centering
    \begin{tabular}{l | l l}
        Sleep modes & Blinking? & Power Consumtion [mW]\\
        \hline
        Idle & Yes & 505 \\
        \hline
        Frozen & Yes & 465 \\
        \hline
        Standby & Yes & 425 \\
        \hline
        Stop & Yes & 425 \\
        \hline
        Deep stop & Yes & 435 \\
        \hline
        Static & No & 435 \\
    \end{tabular}
\end{table}
Since the LED is running on interrupts, as long as there is a clock that
generates them, the LED should blink. In stop and deep stop, the CPU is in sleep
but the RC clock is still running which generates the interrupts. In static,
only an external interrupt can wake the micro-controller.

\section{Power hogs}
\begin{table}[H]
    \centering
    \begin{tabular}{l | l}
        Hardware & Power Consumtion \\
        \hline
        USART & 11.55 \\
        \hline
        All LEDs & 150 \\
        \hline
        LCD & 105
    \end{tabular}
\end{table}

\end{document}
